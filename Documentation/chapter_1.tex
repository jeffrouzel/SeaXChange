%   Filename    : chapter_1.tex 
\chapter{Introduction}
\label{sec:researchdesc}    %labels help you reference sections of your document

\section{Overview}
\label{sec:overview}

The tuna supply chain faces critical issues that affect both the industry and its consumers. It is vulnerable to illegal fishing, overfishing, and lack of traceability, resulting in widespread ethical and environmental concerns. The lack of traceability also contributes to the lack of transparency which results to consumers being compromised by the quality of the product. Through blockchain technology, it can provide solutions to these prevalent challenges by providing a secure and tamper-proof ledger from the tuna's journey from the ocean to plate. This also ensure the compliance of the stakeholders to legal standards.

\noindent Currently, the integration of blockchain with the tuna sector still has its gaps regarding the implementation. This study will help address the industry’s need for transparent and secure tracking of tuna products from ocean to consumer, while assessing the feasibility of implementing blockchain at scale in the seafood sector.


\section{Problem Statement}

Fish is one of the most consumed protein products in the Philippines. Before the COVID-19 pandemic, fish was the most consumed animal protein in the Philippines with annual per capita consumption of 40 kg (Agriculture and Agri-Food Canada, 2022). Although the nation is gifted with an abundance of aquatic resources, the methods of dispersal of the product usually leads to inefficiency in terms of sales, pricing, and overall product quality for the consumers. The problem arises with the introduction of a supply chain from several middlemen between the fish farmer or producer and wholesaler in the coastal and aquatic regions towards the consumers with less access to fresh fish. Consequently, both consumers and suppliers face challenges in ensuring transparent and fair pricing, product tracking, and maintaining the quality of fish products.



\section{Research Objectives}
\label{sec:researchobjectives}

\subsection{General Objective}
\label{sec:generalobjective}

This subsection states the over--all goal that must be achieved to answer the problem.
Address the following: Given your research challenge or opportunity, how do you intend  to solve it? What is the output of your research?


\subsection{Specific Objectives}
\label{sec:specificobjectives}

%
%  \begin{comment} ... \end{comment} is used for multiple lines of comment
%


\begin{enumerate}
	
	\item To create suitable use cases for the tuna supply chain roles, namely, fishers, suppliers, retailers, and consumers through primary sources and acquired information from related literature
	\item To define the system architecture of SeaXChange is also developed in compliance with the necessary blockchain and distributed system architecture to reflect the requirements of the use case
	\item To implement the application after creating the prerequisites of the software development  
	\item To deploy the system after the quality is tested to assess its feasibility in real-world applications 
	\item To discuss the overall results and effectiveness of implementing a blockchain driven application for tuna supply chain in the fourth and fifth chapters of the study
\end{enumerate}


\section{Scope and Limitations of the Research}
\label{sec:scopelimitations}

The scope of this study focuses on how blockchain technology can be applied to enhance traceability and transparency within the tuna supply chain. It will involve features such as smart contracts for recording the transactions and user interface for stakeholders. The study will also focus exclusively on whole, caught tuna products in the supply chain, excluding processed forms such as canned or packaged tuna. The research will examine the traceability of whole tuna from capture to market sale, specifically centering on a supplier based in Iloilo. 

\noindent This study will only be limited to the supply chain in Iloilo, so findings may not fully represent global practices. Since this will only focus on blockchain’s function in traceability, other functions are outside the scope of this research. 


\begin{comment}

%
% IPR acknowledgement: the sentences inside this comment are from Ethel Ong's slides on Scope and Limitations of the Research
%
Generally, one paragraph should be allotted for each of your research objectives.

Each paragraph contains a brief overview of the concept/theory and the purpose of doing the associated objective.

Each paragraph also includes a description of the scope/limitation of your study.

* Please refer to the slides for examples.

\end{comment}


\section{Significance of the Research}
\label{sec:significance}

This study serves a significant purpose for several stakeholders in the tuna supply chain. This study aims to solve the problems related to the management of tuna supply chain, particularly with regards to product traceability.
%STAKEHOLDERS
\begin{itemize}
	\item The Stakeholders 
	\begin{itemize}
		\item This study enhances transparency and accountability which allows stakeholders  such as the fishers, suppliers and retailers to access tamper-proof and accurate information promoting a more ethical and authentic supply chain. Providing a digital record of the product’s history, this study can be beneficial in ensuring the compliance with environmental and legal standards.
	\end{itemize}
\end{itemize}
% CONSUMERS
\begin{itemize}
	\item The Consumers
	\begin{itemize}
		\item Since consumers are now becoming concerned regarding the sustainable sourcing and ethical practices on the products they purchase, this study will be able to help in verifying the history of the tuna product from its origin up until its journey to the consumers, therefore increasing the trust and transparency. 
	\end{itemize}
\end{itemize}
%FUTURE RESEARCHERS
\begin{itemize}
	\item For Future Researchers
	\begin{itemize}
		\item As blockchain technology continues to grow, this study contributes to the application of blockchain in the supply chain management and the insights regarding its benefits and limitations. This research can be helpful in the growing knowledge on digital solutions for traceability and transparency for future research. 
	\end{itemize}
\end{itemize}

\begin{comment}
If applicable, describe possible commercialization and/or innovation in your research.
\end{comment}


