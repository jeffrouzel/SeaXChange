%   Filename    : abstract.tex 
\begin{center}
\textbf{Abstract}
\end{center}
\setlength{\parindent}{0pt}
The tuna supply chain faces critical challenges regarding traceability, transparency and sustainability, particularly due to certain issues such as illegal, unreported and unregulated (IUU) fishing. Ensuring the traceability within the tuna supply chain is a critical role in enhancing consumer confidence, transparency and promoting adherence to environmental and legal standards. This research explores the application of blockchain technology as a solution to these given issues. By combining qualitative insights gathered from different key stakeholders across the supply chain, the researchers evaluated the potential of blockchain to improve product traceability, accountability, and trust. The findings suggests that blockchain offers a secure and transparent method of recording the journey of tuna products from catch to market, helping to combat IUU fishing and promote sustainable practices. However, successful implementation requires overcoming barriers related to technological integration, cost, and stakeholder collaboration. This study provides valuable insights into the feasibility and impact of blockchain adoption within other fish supply chains, contributing to the development of more transparent, responsible, and sustainable tuna industries.

\begin{tabular}{lp{4.25in}}
\hspace{-0.5em}\textbf{Keywords:}\hspace{0.25em} & Blockchain, Traceability, Smart Contract, Supply Chain\\
\end{tabular}
