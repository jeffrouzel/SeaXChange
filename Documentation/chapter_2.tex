%   Filename    : chapter_2.tex 
\chapter{Review of Related Literature}
\label{sec:relatedlit}

In purchasing goods, one thing to consider is the quality of it. An important part of determining the quality is to know the traceability of the supply chain. Traceability refers to the ability of tracking the journey of the product from its source until its destination. The term “traceability” is now more utilized in both the food and production industry (Islam \& Cullen, 2021).  In the context of the tuna supply chain, it can be used not only to promote transparency to consumers but to also ensure compliance with environmental and legal standards. With blockchain technology, the status of tuna at each stage could be recorded in the blockchain which could be used for traceability. This paper aims to address the following research question: \textit{How can blockchain technology improve the traceability of the tuna supply chain management?}

% State of Tuna Industry in the Philippines

\section{State of Tuna Industry in the Philippines}
In 2014, the Philippines became the top global producer of tuna according to Llanto et al (2018). The tuna is caught in domestic and international fishing grounds near the country through various fishing methods such as purse seines, gill nets, handline (hook and line) and ring net. Among the tuna species, the skipjack tuna accounted for the largest portion of the catch by 40\%. The study of PCMARD (1993 as cited in Nepomuceno et al., 2020 ) stated that skipjack tuna are often caught out in open waters or in offshore areas. In addition, Nepomuceno et al. (2020) mentioned in their study that the dominant production of skipjack tuna, together with yellowfin tuna, was recorded in South Cotabato.
The tuna supply has declined since 2000 due to various factors including overfishing, climate change, and the laws and regulations imposed by different governing bodies for the tuna fishing ground such as the Regulation No. 56, released by the Indonesian Maritime Affairs and Fisheries Ministry in November 2014. The regulation imposed a moratorium on issuance of fishing licenses from 3 November 2014 to 30 April 2015 to eliminate illegal, unreported, and unregulated fishing in Indonesian waters near Mindanao where tuna are known to thrive (Llanto et al, 2018). The regulation imposed for the protection of tuna fishing grounds in the western and central pacific ocean also lead to the decline of local tuna production which requires the  fishing operators of the Philippines to invest in the manufacturing and processing of fish particularly tuna in Indonesia which includes hiring Indonesian crew to be deployed in the Philippine fishing vessels (Llanto et al, 2018).

% Fishing Regulations in the Philippines
\section{Fishing Regulations in the Philippines}

A study of Asche et al. (2018) divided the fishing management strategies that include right-based fishery management like territorial use of rights, access rights and harvest rights. It discussed that a rights-based system could support the sustainability of global fisheries by taking in account the three pillars of sustainability (economic development, social development, and environmental protection) rather than focusing on their trade-offs. A restriction on the fisherman’s behavior by harvest rights and catch shares could be a profit problem for them in the short-run but in the long-run, this could help both in the fish stock and the fishermen’s profit. Lack of restriction could lead to overfishing. Access rights limit the entry to fishery through permits which can also reduce the effect of high harvest levels. A sustainable fishing management system in the Philippines is important in order to preserve marine resources. To preserve these resources and protect the livelihood of local communities, various fishing management strategies should be implemented. A collaboration between the fishermen, local government and other stakeholders often happens to manage marine resources (Pomeroy \& Courtney, 2018). The study of Pomeroy and Courtney discussed that marine tenure refers to the rights and responsibilities in terms of who can access the marine and coastal resources. The 1998 Fisheries Code paved the way for local government units (LGUs) to be involved in the management of municipal waters. LGUs are given the responsibility to overlook and regulate fisheries and establish marine tenure rights for fishers within 15 km from shore  and these rights are applicable for municipal fishers and their respective organizations that are listed in the registry (Pomeroy \& Courtney, 2018). In this way, it resolved problems in terms of fishing rights between small-scale and commercial fishing.

According to the study conducted by Mullon et al. (2017), the five major species of tuna: yellowfin \textit{Thunnus albacares}, bigeye \textit{Thunnus obesus}, bluefin \textit{Thunnus thynnus} or \textit{Thunus orientalis}, albacore \textit{Thunnus alalunga}, and skipjack \textit{Katsuwonus pelamis} are harvested to meet the global supply chain demand  which causes those group of tuna fishes to be heavily exploited and threatened. The study conducted by Paillin et al. (2022) states that there are multiple risk agents in the supply chain assessment of tuna, these include the lack of standard environmental management system, lack of maintenance management, and lack of quality control from suppliers. The usage of efficient boats and good quality catching technology can also lead to fisheries depletion which causes various agency such as BFAR (Bureau of Fisheries and Aquatic Resources), the local government units, and the Philippine Coast Guard to enable policies for upholding closed fishing season to restrict large scale fishing vessel to minimize the fishing activities in the identified areas (Macusi et al, 2022). The implementation of closed fishing season caused delay or lack of fish supply, which led to higher fish prices.The growing demands and depleting population of tuna fishes coupled with the rapid increase in fuel costs can have a negative impact on the future of the supply chain in tuna fisheries (Mullon et al., 2017). With factors concerning the slow decline of tuna catches in the Philippines and surrounding nations, the future of the global supply chain of tuna must be addressed.

% Tuna and Fish Supply Chain
\section{Tuna and Fish Supply Chain}

According to Macusi et al (2023), the implementation of traceability programs in the agricultural product commodities and value chain in the Philippines is slower than its competing nation for tuna production. The Philippines has been steadily responding to the market innovation and integration of cost-effective and smart technologies for the traceability of various commodities. Accurate catch data is crucial for determining the attributes of the fish health, size, volumes, and maturity (Grantham et al, 2022) which can be used as a basis for the transparency of the traceability of the fish product. Illegal, unreported, and unregulated (IUU) is another concern for the fish industry. In the 2000s, the persistent IUU became a global crisis affecting the biological, ecological, and socio-economics factors revolving around marine livelihood in Southeast Asia (Malinee et al, 2020). IUU fishing is known to cause short- and long-term problems in the socio-economic opportunities which affects food security and results in the possible collapse of the fish industry and stocks due to overfishing (Malinee et al, 2020). 

The establishment of marine protected areas in the Davao Gulf (MPAs) affected the management of small-scale fisheries due to the growing population and demands for seafood products. The closure of a wide range of fishing areas hosting diverse and marine organisms has affected the socio-economics and livelihood of the local and small-scale fishermen (Macusi et al, 2023), this in turn resulted in IUU fishing.  To ensure that fish stocks in the gulf are sustainably managed, the implementation of GPS for tracking the movement and activities of fishers through logbook and habitat monitoring can provide data and insights for tracking, monitoring, and understanding the condition of the marine resources (Obura et al, 2019; Macusi et al, 2023).

% Tuna Supply Chain Stages and Roles
\section{Tuna Supply Chain Stages and Roles} 

The study conducted by Delfino (2023) highlights the roles of different actors involved in the supply, production, distribution, and marketing of skipjack tuna in Lagonoy Gulf in the Philippines. The study showcased a total of eleven interconnected value chains but are generalized into four major stages or roles - fishers, wholesalers, retailers, and processors. The fishers are the initial players responsible for catching fish using boats or fishing vessels equipped with purse seines, gillnets, and handlines(hook and line). Wholesalers are the actors for selling freshly caught fish locally and regionally, they receive the fish supply directly from the fishers. The next stage after wholesalers are the retailers, these intermediaries sell the fish product to local markets, house-to-house (\textit{libod”} in Visayan languages), and other local medium such as \textit{talipapa} or fish stands. Another intermediary is the processors, they convert fresh skipjack tuna into products like smoked tuna. The given stages also overlapped in some cases as there are fisher-wholesalers who catch and sell the fishes directly to retailers and there are also retailer-processors that both sell whole and processed products. Despite having a firm system to transport fish from sea to table, all the actors face problems during seasonal challenges involving the availability of the tuna product. The fishers also need to consider strict local regulations such as RA 10654 and RA 8550. The strict implementation of RA 10654 and RA 8850 at the local level or the Fisheries Code of the Local Philippines aims to curb the problem encountered during season of deficit tuna supply by limiting fishing activities and implementation of 15-km boundary lines in the municipal waters of each municipality (Delfino, 2023). The study suggests that improving conditions for value chain actors, particularly through support services and government involvement could lead to a stable and sustainable exchange of skipjack tuna and other seafood products from sea to table.

A study of Digal et al. (2017) discussed one of the value chains which was the purse seine or the skipjack tuna value chain in the Philippines. Purse seining is the method of catching a school of fish wherein it uses a large net around it, trapping them and pulling the bottom of the net like a purse-like (Digal et al., 2017). This type of catching often targets dense fishes like tuna. Skipjack fishes that weigh 300 grams and above are often sold  to canneries, while the smaller ones are sold at local markets, often used for consumption by Filipinos. Purse seiners are usually employees of a fishing company and they have a fixed salary. They could reach international waters so they need to bring their passports with them. Jamboleros, who act as distributors, often buy from different fishing companies per \textit{banyera} or tub. They will then pack the fish and sell it to traders/truckers who go to General Santos fishport. These traders will deliver it to the retailers across Kidapawan who contacted them. There is no formal contract between the jambolero and traders/truckers. One of the issues of the retailer is for everyday that a fish is not sold, they would have a \pesos10.00 less per kilogram.

% Factors Affecting the Tuna Supply Chain
\section{Factors Affecting the Tuna Supply Chain}

The tuna supply chain faced several factors and challenges for the safety and quality of the product (Mercogliano \& Santonicola, 2019). Without the proper handling of the tuna after catching it can lead to various food-borne diseases and outbreaks. The most frequent and mitigated food-borne causing compound is Histamine(HIS) which causes Scombrotoxin fish poisoning (SPF) outbreaks related to food allergies when consumed (EFSA, 2017; Peruzy et al., 2017). Tuna species are known for having high presence of amino acid histidine concentrations which are converted to HIS by bacterial enzyme histidine-decarboxylase or HDS (Aponte et al., 2018; Verkhivker \& Altman, 2018). To combat the risk of SPF and other food-borne diseases caused by tuna consumption, several safety hazards and protocols were imposed to the tuna supply chain management. The term cold chain refers to the storing of fish in temperatures less than 5°C after it was caught (Yang \& Lin, 2017). According to the article published by Mercogliano and Santonicola (2019), implementing a cold chain from the time the fish is caught until it is consumed is crucial for mitigating the outbreak of HIS poisoning. Additionally, the article also states that using high-quality raw tuna, cold chain maintenance, pre-cooking, and cooking can also reduce HIS development. 

According to the study conducted by Pacoma and Yap-Dejeto entitled "Health Risk Assessment: Total Mercury in Canned Tuna and in Yellowfin and Frigate Tuna Caught from Leyte Gulf and Philippine Sea", examines mercury contamination in both canned and locally caught tuna in the Philippines. Methylmercury, a potent neurotoxin, presents risks especially to vulnerable groups like pregnant women and children. The study reveals that canned tuna generally has higher mercury levels (0.07 µg/g) than locally caught tuna (0.002–0.024 µg/g). Based on the Food and Agriculture Organization’s fish consumption data, the mercury intake from locally caught tuna is within the World Health Organization’s safe limits, whereas canned tuna may exceed these limits for some groups. This highlights the need for monitoring of the mercury levels in the Philippine tuna supply chain, as tuna is a dietary staple and economic asset in the region, to mitigate health risks from chronic exposure.

Risk management is important for tuna supply chains to analyze the root of the risk and to assess the probability of such cases through the information taken from the different locations or sorting states where the tuna product is handled before being purchased by the consumer or end-use state (Parenreng et al, 2016).

% Technology of Blockchain
\section{Technology of Blockchain}

According to Zheng et al. (2017), the idea of blockchain was first introduced in 2008 and was first implemented in the Bitcoin business which deals with cryptocurrencies. This kind of technology tracks transactions and stores it in a list of blocks. According to Sarmah (2018), it acts as a database of transactions which is overseen and verified by distributed nodes. Blockchain works by linking blocks (where data is stored). When a transaction is initiated, it is then broadcasted to a network of computers that verifies the transaction and if the verification is successful, it will then be grouped and linked with the previous transactions that will be added to the blockchain (Zheng et al., 2017). It does not require a middleman because it operates on a peer-to-peer transaction. This eliminates the traditional way of a central authority like the bank (Sarmah, 2018). 

Automated process of transactions is also one of the salient features of blockchain which is executed automatically based on predefined rules involving no third parties. These predefined rules are conditions that need to be met in order for the transaction to proceed. Given this, blockchain is described to be a “trust-free technology” since it reduces the number of trusted individuals instead they trust the machine itself which is difficult to manipulate given its cryptographic security (Ali et al., 2023). Blockchain also ensures immutability with its data. Once the data has been added to the blockchain, it is difficult to change since each block has a cryptographic link to other blocks, which guarantees that the data is tamper-proof and permanent. (Ali et al., 2023). This also brings back to the essence of blockchain being trustworthy. Immutability and tamper-proof enhances data management. It also helps prevent fraudulent activities especially with finances. Transparency with transactions is one of the features of blockchain which makes the chances of data being tampered less because it is accessible to anyone on the network (Ali et al., 2023).

According to Nasurudeen Ahamed et al. (2020), Blockchain technologies are classified into 3 types: Public Blockchain, Private Blockchain,. Consortium Blockchain. In Public Blockchain, all public peers can join together and have equal rights (for example, read, write, and execute) on the public node. In Private Blockchain, only Authorized Private Peers have access to the network. The access to the node in the private peer is limited to the specific node. In Consortium Blockchain, only the authorized team can access and join this blockchain, and all operations in the node must adhere to the access. Their paper, Sea Food Supply Chain Management Using Blockchain, gave the idea that with the purpose of creating a blockchain-driven application, a public blockchain approach could be appropriate for handling consumer-based information as users can verify non-sensitive data like prices, freshness and availability. While handling sensitive information such as internal works and logistics, a consortium blockchain where authorized users such as fish owners, distributors, manufacturers, etc. can handle the core supply chain operations, like tracking the movement of tuna from catch to market.


% Opportunities of Blockchain Technology for Supply Chain Management 
\section{Opportunities of Blockchain Technology for Supply Chain Management}

Supply chain is the term used for understanding the business activities for designing, developing, delivering, purchasing, and using a product or service (Hugos, 2024). Companies and various industries are heavily relying on supply chains to achieve their business objectives. The purpose of supply chain began to be more significant in the last century as firms discovered that supply chain can be used for competitive advantage instead of just a cost driver as believed in the bygone days (Snyder \& Shen, 2019). Following the supply chain paradigm can demonstrate the delivery of a product or service while strongly emphasizing the customer’s specifications. With the increasing studies conducted and published for supply chain, many companies adopted this practice for the benefit of their longevity,  as such the term supply chain management has come into place. The Council of Supply Management Professionals or CSCMP (2024) defines supply chain management as the planning and management of all activities involved in sourcing and procurement, conversion, and all logistics management activities; essentially, supply chain management integrates supply and demand management within and across the company. Supply chain management is also involved with the relationship with collaborators and channel partners such as suppliers, intermediaries, third party providers, and customers (CSCMP, 2024).

In the article of Cordova et. al (Cordova et. al, 2021), the role of supply chain management and the growing opportunities for blockchain technology in supply chain management was discussed. According to Cordova (2021), the recent innovation and globalization has given rise to the idea of using a data innovation framework for supply chain management. Technologies such as blockchains and enterprise resource planning (ERP) are among the highly contested platforms for supply chain management to operate in a seamless interaction and distribution with the product while heavily relying on modern technology and innovations. The logistic business of the supply chain market is wide and complex, the distribution and flow of products is not a simple job, and it heavily relies on paperwork (Georgiou, 2019;  Cordova et. al, 2021). The usage of paperwork for logistic business can be at higher risk for lack of transparency, complex or unreliable tracking, deficiency of information, and possible dispute due to the tendency of paper to disappear or tear down, this can in turn delay the process and delivery of the item/product. With the issues encountered in the supply chain market, businesses, people and enterprises are eyeing toward the application of blockchain technology on supply chain management (Cordova et. al, 2021).

Implementing blockchain innovation in ERP systems and companies that use digital platforms can  provide opportunities and contribute greatly for business processes (EOS Costa Rica, 2019 as cited by Cordova et al, 2021).  The ability of blockchain technology to append new transactions to an existing block containing data can be thought of as a decentralized ledger (Cole et al, 2019). The method of blockchain to behave like a decentralized ledger can serve as a single unified source of data which in turns create a clear and consistent audit trail involved in the manufacturing, assembly, supply, and maintenance processes. According to Cole et al (2019), blockchains provide data to the movement and relation of products from its origin, inventory, shipment, and purchase. One potential of blockchain for supply chain management (Niels \& Moritz, 2017) is the ease of paperwork processing, specifically in ocean freight. When IBM and Maersk settled for a permissioned blockchain solution, they were able to connect a global network of shippers, carriers, ports, and customs. Another potential of blockchain in SCM is to identify counterfeit products. In the pharmaceutical industry and healthcare setting, blockchain could improve patient safety and hazard through establishing supply chain transparency from manufacturers through wholesale and pharmacies to the individual patients (Niels \& Moritz, 2017). Using blockchain can make it harder to tamper or alter the products chain of origin with illegal and counterfeit products. Blockchain have the potential to facilitate origin tracking. According to Cordova et al (2021), blockchain allows organizations to input relevant data inside a chain which would have constant updates and tracking, this supports visibility and traceability of the origin of the product. Smart contracts, an executable code and a feature of blockchain, serves as a computer protocol made between participants to digitally facilitate, execute, verify, and enforce an agreement or terms of contract which is then stored in the blockchain (Khan et al, 2021).

% Supply Chain Model with Blockchain Technology of Fishing Industry in Indonesia
\section{Supply Chain Model with Blockchain Technology of Fishing Industry in Indonesia}

Larissa and Parung (2021) who explored the application of blockchain and designed a supply chain model based on it, specifically for the Indonesian fishing industry, aimed to mitigate the challenges in the fishery industry such as product quality (perishability), long shipping times ,and data manipulation. The model they developed of using QR codes for each player in the supply chain then tracking it by scanning the QR code, could inspire our approach in building a blockchain-driven application for the tuna supply chain in the Philippines.

% Existing Technology Intended for Traceability and Supply Chain
\section{Existing Technology Intended for Traceability and Supply Chain}

A study of Shamsuzzoha et al.  (2023) discussed the feasibility of implementing a blockchain driven application called ‘Tracey’ for monitoring the fish traceability in supply chain management. The study utilized the theoretical framework developed by Islam \& Cullen (2021) for improving the understanding and effectiveness of implementing a food traceability system. The framework consists of four principles as a basis for the supply chain management: identification, data recording, data integration, and accessibility (Islam \& Cullen, 2021). The Tracey application utilized a public-private hybrid blockchain-based conceptual framework by Mantravadi and Srai (2023) to uphold the transparency, traceability, and certification of the sea food produce, specifically shrimp. The prototype being studied by Shamsuzzoha et al. (2023) called Tracey focuses on the mobile-based solution approach, the study found that the most widely used smartphone type in the Philippines is the android phone which is where the Tracey prototype is intended to be used. The Tracey app allows fishermen to log their catch details and buyers to verify and update transaction history (Shamsuzzoha et al., 2023). The Tracey app uses a central database for storing fish trading data and a decentralized ledger or blockchain for traceability purposes. The decentralized ledger acts as a tamper-proof copy of the data recorded by fishermen and buyers. The result of the study finds that fishermen are open to using digital methods for payments and confidentiality which is required for exporting the fish product to maintain high standards for traceability, catch certification, and product quality. The usage of blockchain as exemplified by the Tracey project can be used for upholding the restriction for IUU due to its ability to ensure transparent trade, consistent records, and accessibility. The result and discussion of the study of Shamsuzzoha et al. (2023) provides a solution for improving the sustainability of tuna fishery and ensures that Filipino fishermen receive fair compensation. For the study limitation of the Tracey project, although there is a high acceptability of potentially using the app for fishermen, there are still constraints in terms of proper incentives, connectivity issues, technology usability, and education for using the app.

A study of Cocco and Mannaro (2021) proposed a blockchain-based technology in the traceability of the supply chain management of a traditional Italian food product, Carasau bread, which is made from durum wheat flour, salt, yeast and water. Since the production of this product is traditional, consumers would demand for transparency on the methods used in the Carasau bread production to ensure authenticity. The proposed model involves the combination of the application of Internet of Things (IoT), specifically the Radio Frequency Identification (RFID) sensors and Interplanetary File System (IPFS) with Blockchain (Cocco \& Mannaro, 2021). RFID is a technology that uses radio frequencies to identify and track a tagged object while IPFS allows files to be stored and tracked over a decentralized and distributed file system. Cocco and Mannaro (2021) also stated that using RFID tags that will be integrated with different sensors to monitor food quality will be useful in their study. These sensors include freshness indicators to monitor the food quality when packaged, biosensors to detect degradation molecules, time temperature indicator to measure and record temperature and humidity sensors to detect the amount of water vapor in the atmosphere. The integration of IoT and blockchain technology can have a huge impact in increasing traceability in agri-food supply chain. Moreover, this study proposed to have a generic agri-food traceability system which will be based on Ethereum blockchain, Radio-frequency identification (RFID), Near Field Communication (NFC), and Interplanetary File System (IPFS) technology. Moreover, the model proposal also includes sensor network devices, smart contacts, optical cameras and an external database. Each IoT device will be connected to Raspberry Pi and interfaces with blockchain implementing smart contracts and IPFS which authorities can inspect every node and batch online through the uploaded files in IFPS; hashes of the uploaded files on IFPS are also stored on blockchain (Cocco \& Mannaro, 2021). In this way, users along the chain supply can view and trace each batch using the NFC tags promoting transparency and traceability. Overall, the proposed model is a combination of two subsystems. The first one is an on-chain system which is the blockchain implementing smart contracts and will be developed using Solidity, an object-oriented language. The second one is an off-chain system that will be implemented in Javascript using Node.js (to interact with the smart contracts) and Web3.js packages (to interact with blockchain) and these packages should be installed on the Raspberry Pi. However, the purpose of this study is to examine the traceability systems of the agri-food industry and further provide possible solutions. 

% Developing a Traceability System for Tuna Supply Chains
\section{Developing a Traceability System for Tuna Supply Chains}

The study of Kresna et al. (2017), proposed an IT-based traceability system for tuna supply chain as opposed to the traditional paper based traceability system which has several limitations such as the potency to be manipulated, error by the human, language barrier, and physical damage. The architecture comprises several layers: infrastructure, data, application, communication, and user layers. The infrastructure layer includes computer hardware, network infrastructure, and sensing devices like CCTV, GPS, and RFID for data acquisition. The data layer serves as the system's database, featuring both a main system database and an emergency database for critical situations. The application layer consists of various modules—admin, tracing, transporter, supplier, and government—that registered actors can access through different interfaces. Finally, the user layer consists of the registered actors who utilize the system.

The journal article of Tiwari (2020) called Application of Blockchain in Agri-Food Supply Chain conducted two case studies for a blockchain driven app built for supply chain related to food, fishing, and agriculture. The first case-study is the usage and effectiveness of the Provenance system for tuna tracking certification. The objective of the Provenance system is to enhance transparency in the tuna supply chain by ensuring certification and standard compliance across all roles(e.g. supplier, retailer) in the chain. The system is built using six modular programs: registering, standards, production, manufacturing, tagging, and user-interface. The usage of blockchain in the Provenance system allows transactions to be recorded to allow  shared ledger for transparency and smart contracts for secure exchanges of money or information. The usage of the Provenance system is to solve the issues encountered in the tuna fishing industry affected by various factors such as illegal, unregulated, unauthorized (IUU) fishing, fraud, and human rights abuses. The solution of the Provenance system is to allow tracking, tracing, and certification of tuna using blockchain. The Provenance system has a smart tagging feature that allows fishermen to use SMS for digital assets on the blockchain to track where the fish, in return, all supply chain stakeholders can access the data that was sourced from the SMS. The second case-study is the usage of the IBM Food Trust for transparency in the food supply chain. The IBM Food Trust aims to solve the problems in the food supply chain, specifically in product safety. Locating supply chain items in real-time using identifiers like GTIN or UPC is the primary feature of the IBM Food Trust. The app also provides end-to-end product provenance, real-time location and status, and facilitates rapid product recalls. The IBM Food Trust also provides insights and visibility for the freshness of the product to reduce losses and spoilage. Lastly, the IBM Food Trust provides certifications from the information taken when handling and managing the products in the supply chain. The case studies conducted by Tiwari (2020) illustrates the potential of blockchain technology in improving transparency, efficiency, and ethical practices within supply chains.

\section{Chapter Summary}

\subsection{Gaps}
Given the advanced existing technologies in blockchain-based traceability systems for agri-food supply chains, significant gaps remain in understanding the user experience and integration challenges faced by the fishermen. Since the study of Shamsuzzoha et al.(2023) mentioned the feasibility and benefits of mobile application, Tracey, they somehow overlooked the possible problems like technology adoption, digital literacy and issues of connectivity. Moreover, existing technologies also focused on large-scale implementations and theoretical frameworks without considering the practical implications and user experience necessary for the effective integration of the system. This study aims to address these gaps by exploring real-world challenges faced by the users, especially by the fishermen, in adopting blockchain technology for traceability.

\subsection{Summary}
The literature reviewed highlighted the critical challenges and opportunities regarding the tuna supply chain. It highlighted the issues of traceability and sustainability. There were some existing supply chain technologies which presented solutions, particularly using blockchain technology but it also has its limitations in blockchain adoption. Application of blockchain in the tuna supply chain has shown potential in promoting and improving traceability from ocean to consumers. Through this paper, a blockchain-driven solution could contribute in providing a more efficient and transparent supply chain. Moreover, further studies still remain in assessing the sustainability of blockchain in the long-run.













