%   Filename    : chapter_4.tex 
\chapter{Research Methodology}
In this chapter, it outlines a clear and detailed description of the research methods and processes used in the development and evaluation of SeaExchange: A Blockchain Driven App for Tuna Supply Chain Management. The algorithms, systems, theories, framework and models are described in detail in which this chapter establishes the foundation of this study .This chapter also explains the data collection method used ensuring the validity and reliability of the results.In addition, the chapter discusses the considerations and potential limitations of this study. Overall, this will serve as a guide for the readers in understanding the structured process of developing the SeaXChange.

\section{Research Activities}
For this study, the researchers opted for an interview because it enables in-depth exploration of stakeholder perspectives and experiences. 
The identified fisher and supplier client interface will be tested in the perimeters of Jagnee Fishing Corporation in Tiolas, San Joaquin, Iloilo, Philippines. The identified retailers will be the vendors who sourced their tuna products from Jagnee Fishing Corporation. The identified retailer and consumer testers are situated in the Miagao and San Joaquin areas. 


\begin{itemize}
   \item who is responsible for the task
   \item the resource person to be contacted
   \item what will be done
   \item when and how long will the activity be done
   \item where will it be done
   \item why should be activity be done
\end{itemize}

\textcolor{red}{DO NOT FORGET to cite your references.}

\section{Calendar of Activities}

A Gantt chart showing the schedule of the activities should be included as a table. For example:

Table \ref{tab:timetableactivities} shows a Gantt chart of the activities.  Each bullet represents approximately
one week worth of activity.

%
%  the following commands will be used for filling up the bullets in the Gantt chart
%
\newcommand{\weekone}{\textbullet}
\newcommand{\weektwo}{\textbullet \textbullet}
\newcommand{\weekthree}{\textbullet \textbullet \textbullet}
\newcommand{\weekfour}{\textbullet \textbullet \textbullet \textbullet}

%
%  alternative to bullet is a star 
%
\begin{comment}
   \newcommand{\weekone}{$\star$}
   \newcommand{\weektwo}{$\star \star$}
   \newcommand{\weekthree}{$\star \star \star$}
   \newcommand{\weekfour}{$\star \star \star \star$ }
\end{comment}



%   Filename    : chapter_4.tex 
\chapter{Research Methodology}
In this chapter, it outlines a clear and detailed description of the research methods and processes used in the development and evaluation of SeaExchange: A Blockchain Driven App for Tuna Supply Chain Management. The algorithms, systems, theories, framework and models are described in detail in which this chapter establishes the foundation of this study .This chapter also explains the data collection method used ensuring the validity and reliability of the results.In addition, the chapter discusses the considerations and potential limitations of this study. Overall, this will serve as a guide for the readers in understanding the structured process of developing the SeaXChange.

\section{Research Activities}
For this study, the researchers opted for an interview because it enables in-depth exploration of stakeholder perspectives and experiences. 
The identified fisher and supplier client interface will be tested in the perimeters of Jagnee Fishing Corporation in Tiolas, San Joaquin, Iloilo, Philippines. The identified retailers will be the vendors who sourced their tuna products from Jagnee Fishing Corporation. The identified retailer and consumer testers are situated in the Miagao and San Joaquin areas. 



\section{Calendar of Activities}
%
%  the following commands will be used for filling up the bullets in the Gantt chart
%
\newcommand{\weekone}{\textbullet}
\newcommand{\weektwo}{\textbullet \textbullet}
\newcommand{\weekthree}{\textbullet \textbullet \textbullet}
\newcommand{\weekfour}{\textbullet \textbullet \textbullet \textbullet}

%
%  alternative to bullet is a star 
%

\begin{table}[H]
	\centering
	\caption{Timetable of Activities} \vspace{0.25em}
	\resizebox{\textwidth}{!}{%
		\begin{tabular}{|>{\centering\arraybackslash}p{1.5in}|c|c|c|c|c|c|c|c|c|c|c|c|}
			\hline
			Activities (2024) & Aug & Sep & Oct & Nov & Dec & Jan & Feb & Mar & Apr & May & Jun \\ \hline
			Brainstorming and Selection of Topic   & \weekone & \weektwo &  &  &  &  &  &  &  &  &  \\ \hline
			Review of Related Literature & \weekthree & \weekfour &  &  &  &  &  &  &  &  &  \\ \hline
			Interview Potential Stakeholders      &  & \weektwo  & \weekfour & \weekone &  &  &  &  &  &  &  \\ \hline
			Proposal Document Creation in LaTex     &   &  &  & \weekfour & \weekfour &  &  &  &  &  &  \\ \hline
			Mockups and Prototype      &   &  &  & \weektwo & \weekfour &  &  &  &  &  &  \\ \hline
			Proposal Presentation &   &  &  &  & \weekone &  &  &  &  &  &  \\ \hline
			Development and Testing of Software &  &  &  &  &  & \weekfour & \weekfour & \weekfour & \weekfour & \weekfour & \weekone \\ \hline
			Deployment of Software &  &  &  &  &  &  &  &  &  &  & \weekone \\ \hline
			Results and Feedback  &  &  &  &  &  &  &  &  &  &  & \weekone \\ \hline
		\end{tabular}%
	}
	\label{tab:timetableactivities}
\end{table}

