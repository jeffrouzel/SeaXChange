%   Filename    : acknowledgment.tex 
\begin{center}
	\textbf{Acknowledgment}
\end{center}

Our deepest appreciation goes to our thesis adviser,\textbf{Francis D. Dimzon, Ph.D.}, for his expertise, patient guidance, and unwavering support. We are also grateful to \textbf{Christi Florence C. Cala-or}, panelist, for her insightful feedback and suggestions that enriched this study.

We thank the \textbf{University of the Philippines Visayas}, particularly the \textbf{Division of Physical Sciences and Mathematics}, College of Arts and Sciences, for providing the academic foundation and resources. We also acknowledge the \textbf{Computer Science Faculty} for equipping us with essential skills in software engineering.

Special thanks are extended to the \textbf{Institute of Marine Fisheries and Oceanology}. We particularly recognize \textbf{Ricardo P. Babaran Ph.D.}, , for his valuable insights into the state of the tuna supply chain, existing technologies, and the importance of supporting marginalized groups within it. We also thank \textbf{Carmelo Del Castillo, Ph.D.}, for sharing related insights from prior thesis work on the tuna supply chain. 

To our fellow Special Problem classmates and colleagues—thank you for the shared learning experiences, collaboration, and mutual support that made this academic journey more fulfilling.

The successful implementation of SeaXChange was made possible through the use of several tools and platforms. We acknowledge \textbf{Google Cloud Platform} for hosting our backend services and blockchain network, and the \textbf{Hyperledger Fabric} open-source framework, along with its documentation and community, for providing a robust foundation for our blockchain system.

We extend our gratitude to the community participants and organizations in \textbf{Miagao} who generously shared their time and perspectives. Our sincere thanks go to the respondents—fisherfolk, suppliers, retailers, and consumers—whose real-world experiences were vital to this study. We especially thank \textbf{Jerome F. Cabatuan} and \textbf{Veronica Jeruta} for facilitating access and participation within their respective communities. We also acknowledge \textbf{Jagnee Fishing Corporation} and \textbf{Engr. Noel Lucero} for sharing their knowledge of the fish industry, fishing vessels, and the tuna supply chain.

Finally, and most importantly, we extend our heartfelt thanks to our families for their unwavering love, encouragement, sacrifices, and patience throughout the duration of this study.

This thesis is a culmination of the support and collaboration of all those mentioned. We hope that this work contributes meaningfully to improving transparency and efficiency in the tuna supply chain.
