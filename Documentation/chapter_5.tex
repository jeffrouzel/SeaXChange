%   Filename    : chapter_4.tex 
\chapter{Conclusion and Recommendations}
\section{Overview}
This chapter  presents the key findings and conclusion of the study. It also discusses how the system addresses the challenges of traceability within the tuna supply chain. This chapter also provides recommendations in enhancing the system's functionality and usability, ensuring that the SeaXChange app continues to meet evolving needs of its users. 


\section{Conclusion}
This study aimed to develop and evaluate SeaXChange, which is a blockchain-driven app designed to enhance transparency, traceability and accountability within the tuna supply chain. Through the adaption of Scrum, the team was able to develop a functional prototype that was based on iterative development to achieve goals. 

\noindent The results from the gathered data suggests that the app has effectively addressed key challenges in traceability and accountability, especially among suppliers and consumers. The stakeholders consistently rated the system as good. However, some areas need improvement, especially in ensuring seamless interaction and data security for fishermen.

\noindent Overall, SeaXChange demonstrates strong potential as a technological solution for promoting transparency, combating illegal, unreported, and unregulated (IUU) fishing, and empowering stakeholders across the tuna supply chain.

\section{Recommendations}
After analyzing and interpreting the gathered data, the researchers had identified some improvements and recommendations for the further development and implementation of the SeaXChange app.
	\begin{enumerate}
		\item \textbf{Incorporation of Local Language}
		\\Since most of the target users are within Miagao, the system could provide multi-language support, including the Karay-a language. This will improve guidance through clearer button descriptions.
		
		\item \textbf{Utilization of IoT}
		\\The system could use Internet of Things (IoT) in verifying the fisherman's location. This will add more accountability in tracing the fisherman's current location.
		
		\item \textbf{Inclusion of User Manual }
		\\To further enhance the experience of its users, the researcher could provide printed or digital copies on the system's functionalities. This will help users navigate through the system without being lost.
	\end{enumerate}
	
	
\noindent In conclusion, the development of the SeaXChange app highlights the critical role of emerging technologies in providing solution regarding the traceability, transparency and accountability within the tuna supply chain. While the system has demonstrated strong potential, continuous improvements are still needed to ensure its effectiveness. Moreover, further development and usability enhancements will be essential in attaining SeaXChange's goal of creating a more traceable, transparent and accountable tuna industry.


