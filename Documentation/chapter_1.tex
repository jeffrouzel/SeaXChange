%   Filename    : chapter_1.tex 
\chapter{Introduction}
\label{sec:researchdesc}    %labels help you reference sections of your document

\section{Overview}
\label{sec:overview}

The tuna supply chain faces critical issues that affect both the industry and its consumers. It is vulnerable to illegal fishing, overfishing, and lack of traceability, resulting in widespread ethical and environmental concerns. The lack of traceability also contributes to the lack of transparency which results to consumers being compromised by the quality of the product. Through blockchain technology, it can provide solutions to these prevalent challenges by providing a secure and tamper-proof ledger from the tuna's journey from the ocean to plate. This also ensure the compliance of the stakeholders to legal standards.

\noindent Currently, the integration of blockchain with the tuna sector still has its gaps regarding the implementation. This study will help address the industry’s need for transparent and secure tracking of tuna products from ocean to consumer, while assessing the feasibility of implementing blockchain at scale in the seafood sector.


\section{Problem Statement}

Fish is one of the most consumed protein products in the Philippines. Before the COVID-19 pandemic, fish was the most consumed animal protein in the Philippines with annual per capita consumption of 40 kg (Agriculture and Agri-Food Canada, 2022). Among the wide variety of fish, tuna stands out as a particularly significant commodity due to its high demand and economic value. Although the nation is gifted with an abundance of aquatic resources, the methods of dispersal of the product usually leads to inefficiency in terms of sales, pricing, and overall product quality for the consumers. The problem arises with the introduction of a supply chain from several middlemen between the fish farmer or producer and wholesaler in the coastal and aquatic regions towards the consumers with less access to fresh fish. Consequently, both consumers and suppliers face challenges in ensuring transparent and fair pricing, product tracking, and maintaining the quality of fish products.



\section{Research Objectives}
\label{sec:researchobjectives}

\subsection{General Objective}
\label{sec:generalobjective}

The general objective of the study is to design and develop a blockchain-driven application that would help to improve the traceability of the tuna supply chain. Given the timely issues regarding illegal, unreported and unregulated (IUU) fishing and lack of product traceability and transparency,  this study tends to address these challenges through an application that has the capability to provide an immutable ledger and tamper-proof records. The result of this study will serve as a framework in integrating blockchain technology in the fish supply chain, specifically tuna. This would also support future researchers and developers with similar challenges in the fish supply chain.


\subsection{Specific Objectives}
\label{sec:specificobjectives}

To further specify the research objectives, the study focuses on the following activities. 

\begin{enumerate}
	
	\item To develop a smart contract framework using blockchain technology for data verification and transaction recording, ensuring secure and tamper-proof data for the stakeholders
	\item To design and develop a blockchain-driven application with a user-friendly interface that allows stakeholders to access and input data while enhancing traceability in the tuna supply chain through a tuna asset record for the supply chain participants.
	\item To deploy the application after completing all necessary preparations for software development and to evaluate its overall results and effectiveness in enhancing the tuna supply chain, as discussed in the fourth and fifth chapters of the study.
\end{enumerate}


\section{Scope and Limitations of the Research}
\label{sec:scopelimitations}

The scope of this study focuses on how blockchain technology can be applied to enhance traceability and transparency within the tuna supply chain. It will involve features such as smart contracts for recording the transactions and user interface for stakeholders. The study will also focus exclusively on whole, small-sized caught tuna products in the supply chain, excluding processed forms such as canned or packaged tuna. The research will examine the traceability of whole tuna from capture to market sale, specifically centering on a supplier based in Miagao and San Joaquin, Iloilo. 

\noindent This study will only be limited to the supply chain in specified municipalities of Iloilo, so findings may not fully represent global practices. Since this will only focus on blockchain’s function in traceability, other functions are outside the scope of this research. 


\begin{comment}

%
% IPR acknowledgement: the sentences inside this comment are from Ethel Ong's slides on Scope and Limitations of the Research
%
Generally, one paragraph should be allotted for each of your research objectives.

Each paragraph contains a brief overview of the concept/theory and the purpose of doing the associated objective.

Each paragraph also includes a description of the scope/limitation of your study.

* Please refer to the slides for examples.

\end{comment}


\section{Significance of the Research}
\label{sec:significance}

This study serves a significant purpose for several stakeholders in the tuna supply chain. This study aims to solve the problems related to the management of tuna supply chain, particularly with regards to product traceability.
%STAKEHOLDERS
\begin{itemize}
	\item The Stakeholders 
	\begin{itemize}
		\item This study enhances transparency and accountability, allowing stakeholders such as fishers, suppliers, and retailers to access tamper-proof and accurate information, thereby promoting a more ethical and authentic supply chain. By providing a digital record of the product’s history, this study helps ensure compliance with environmental and legal standards. In cases of anomalies such as oil spills, red tide occurrences, and illegal fishing activities, stakeholders can be involved in identifying and addressing these issues, fostering a collaborative approach to sustainability. Similarly, the record of a tuna asset can be utilized for accountability purposes when problems such as damaged products, mislabeling, or contamination arise, allowing stakeholders to trace and resolve them efficiently.
	\end{itemize}
\end{itemize}
% CONSUMERS
\begin{itemize}
	\item The Consumers
	\begin{itemize}
		\item This study will be able to help consumers verify the history of the tuna product from its origin up until its journey to the consumers, therefore increasing trust and transparency. By promoting traceability, it provides a more detailed and verifiable record of the supply chain, enabling consumers to assess sustainability practices and identify stakeholders responsible for any potential issues affecting the tuna product. This, in turn, encourages critical evaluation of the tuna supply chain, driving improvements in accountability, resource management, and ethical sourcing.
	\end{itemize}
\end{itemize}
%FUTURE RESEARCHERS
\begin{itemize}
	\item For Future Researchers
	\begin{itemize}
		\item As blockchain technology continues to grow, this study contributes to the application of blockchain in the supply chain management and the insights regarding its benefits and limitations. This research can be helpful in the growing knowledge on digital solutions for traceability and transparency for future research. 
	\end{itemize}
\end{itemize}

\begin{comment}
If applicable, describe possible commercialization and/or innovation in your research.
\end{comment}


