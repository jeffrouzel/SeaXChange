%%%%%%%%%%%%%%%%%%%%%%%%%%%%%%%%%%%%%%%%%%%%%%%%%%%%%%%%%%%%%%%%%%%%%%%%%%%%%%%%%%%%%%%%%%%%%%%%%%%%%%
%
%   Filename    : appendix_A.tex 
%
%   Description : This file is for including the Research Ethics Documents (delegated as Appendix A) 
%                 
%%%%%%%%%%%%%%%%%%%%%%%%%%%%%%%%%%%%%%%%%%%%%%%%%%%%%%%%%%%%%%%%%%%%%%%%%%%%%%%%%%%%%%%%%%%%%%%%%%%%%%

\chapter{Code Snippets}
\label{sec:appendixa}

\begin{lstlisting}[style=customjs, caption={Role-based access logic for tuna asset visibility}]
	const checkAssetAccess = (
	asset, userIdentifier, role
	) => {
		switch (role.toLowerCase()) {
			case 'fisher':
			if (asset.Fisher === userIdentifier)
			return { hasAccess: true, accessType: 'full' };
			break;
			case 'supplier':
			if (asset.Supplier === userIdentifier)
			return { hasAccess: true, accessType: 'full' };
			break;
			case 'retailer':
			if (asset.Retailers?.includes(userIdentifier))
			return { hasAccess: true, accessType: 'full' };
			break;
			case 'consumer':
			if (asset.Consumers?.includes(userIdentifier))
			return { hasAccess: true, accessType: 'full' };
			break;
		}
		
		if (role.toLowerCase() === 'consumer') {
			return { hasAccess: true, accessType: 'readonly' };
		}
		
		return { hasAccess: false, accessType: 'none' };
	};
\end{lstlisting}

\begin{lstlisting}[style=customjs, caption={Node.js configuration for Hyperledger Fabric Gateway}]
	const grpc = require('@grpc/grpc-js');
	const { connect, hash, signers } = require('@hyperledger/fabric-gateway');
	const crypto = require('node:crypto');
	const fs = require('node:fs/promises');
	const path = require('node:path');
	const { TextDecoder } = require('node:util');
	
	const channelName = envOrDefault('CHANNEL_NAME', 'mychannel');
	const chaincodeName = envOrDefault('CHAINCODE_NAME', 'basic');
	const mspId = envOrDefault('MSP_ID', 'Org1MSP');
	
	// Path to crypto materials.
	const cryptoPath = envOrDefault(
	'CRYPTO_PATH',
	path.resolve(
	__dirname,
	'..',
	'..',
	'..',
	'test-network',
	'organizations',
	'peerOrganizations',
	'org1.example.com'
	)
	);
	
	const keyDirectoryPath = envOrDefault(
	'KEY_DIRECTORY_PATH',
	path.resolve(
	cryptoPath,
	'users',
	'User1@org1.example.com',
	'msp',
	'keystore'
	)
	);
	
	const certDirectoryPath = envOrDefault(
	'CERT_DIRECTORY_PATH',
	path.resolve(
	cryptoPath,
	'users',
	'User1@org1.example.com',
	'msp',
	'signcerts'
	)
	);
	
	const tlsCertPath = envOrDefault(
	'TLS_CERT_PATH',
	path.resolve(cryptoPath, 'peers', 'peer0.org1.example.com', 'tls', 'ca.crt')
	);
	
	const peerEndpoint = envOrDefault('PEER_ENDPOINT', 'localhost:7051');
	const peerHostAlias = envOrDefault('PEER_HOST_ALIAS', 'peer0.org1.example.com');
	
	const utf8Decoder = new TextDecoder();
	const assetId = asset${String(Date.now())};
\end{lstlisting}


% Asset Struct Definition
\begin{lstlisting}[style=customjs, caption={Asset Data Structure}, label={lst:asset-struct}]
	type Asset struct {
		ID             					string `json:"ID"`
		Species							string `json:"Species"`
		Weight							float64 `json:"Weight"`
		CatchLocation					string `json:"CatchLocation"`
		CatchDate						string `json:"CatchDate"`
		FishingMethod					string `json:"FishingMethod"`   
		Fisher							string `json:"Fisher"`
		Supplier						string `json:"Supplier"`
		SellingLocationSupplier			string `json:"SellingLocationSupplier"`
		Retailers						[]string `json:"Retailers"`
		SellingLocationRetailers		[]string `json:"SellingLocationRetailers"`
		Consumers						[]string `json:"Consumers"`
	}
\end{lstlisting}

% CreateAsset Function
\begin{lstlisting}[style=customjs, caption={CreateAsset Function}, label={lst:create-asset}]
	func (s *SmartContract) CreateAsset(ctx contractapi.TransactionContextInterface, id string, species string, weight float64, catchLocation string, catchDate string, fishingMethod string, fisher string) error {
		exists, err := s.AssetExists(ctx, id)
		if err != nil {
			return err
		}
		if exists {
			return fmt.Errorf("the asset %s already exists", id)
		}
		
		asset := Asset{
			ID:             id,
			Species:		species,
			Weight:			weight,
			CatchLocation:	catchLocation,
			CatchDate:		catchDate,
			FishingMethod:	fishingMethod,
			Fisher:			fisher,
			Retailers:		[]string{},
			SellingLocationRetailers:	[]string{},
			Consumers:      []string{},
		}
		assetJSON, err := json.Marshal(asset)
		if err != nil {
			return err
		}
		
		return ctx.GetStub().PutState(id, assetJSON)
	}
\end{lstlisting}

% TransferAsset Function
\begin{lstlisting}[style=customjs, caption={TransferAsset Function}, label={lst:transfer-asset}]
	func (s *SmartContract) TransferAsset(ctx contractapi.TransactionContextInterface, id string, role string, newParticipant string, newLocation string) (string, error) {
		asset, err := s.ReadAsset(ctx, id)
		if err != nil {
			return "", fmt.Errorf("failed to fetch asset: %v", err)
		}
		
		switch role{
			case "Supplier":
			oldSupplier := asset.Supplier
			asset.Supplier = newParticipant
			asset.SellingLocationSupplier = newLocation
			return oldSupplier, s.SaveAsset(ctx, id, asset)
			case "Retailer":
			if !contains(asset.Retailers, newParticipant){
				asset.Retailers = append(asset.Retailers, newParticipant)
				asset.SellingLocationRetailers = append(asset.SellingLocationRetailers, newLocation)
			}
			return "", s.SaveAsset(ctx, id, asset)
			case "Consumer":
			if !contains(asset.Consumers, newParticipant){
				asset.Consumers = append(asset.Consumers, newParticipant)
			}
			return "", s.SaveAsset(ctx, id, asset)
			default:
			return "", fmt.Errorf("invalid role specified: %s", role)
		}
	}
\end{lstlisting}

% ReadAsset Function
\begin{lstlisting}[style=customjs, caption={ReadAsset Function}, label={lst:read-asset}]
	func (s *SmartContract) ReadAsset(ctx contractapi.TransactionContextInterface, id string) (*Asset, error) {
		assetJSON, err := ctx.GetStub().GetState(id)
		if err != nil {
			return nil, fmt.Errorf("failed to read from world state: %v", err)
		}
		if assetJSON == nil {
			return nil, fmt.Errorf("the asset %s does not exist", id)
		}
		var asset Asset
		err = json.Unmarshal(assetJSON, &asset)
		if err != nil {
			return nil, err
		}
		if asset.Consumers == nil {
			asset.Consumers = []string{}
		}	
		
		return &asset, nil
	}
\end{lstlisting}

% GetAllAssets Function
\begin{lstlisting}[style=customjs, caption={GetAllAssets Function}, label={lst:get-all-assets}]
	func (s *SmartContract) GetAllAssets(ctx contractapi.TransactionContextInterface) ([]*Asset, error) {
		resultsIterator, err := ctx.GetStub().GetStateByRange("", "")
		if err != nil {
			return nil, err
		}
		defer resultsIterator.Close()
		
		var assets []*Asset
		for resultsIterator.HasNext() {
			queryResponse, err := resultsIterator.Next()
			if err != nil {
				return nil, err
			}
			
			var asset Asset
			err = json.Unmarshal(queryResponse.Value, &asset)
			if err != nil {
				return nil, err
			}
			if asset.Consumers == nil {
				asset.Consumers = []string{}
			}
			assets = append(assets, &asset)
		}
		
		return assets, nil
	}
\end{lstlisting}


\begin{lstlisting}[style=customjs, caption={\texttt{main()} Function and Gateway Logic}, label={lst:main-gateway}]
	async function main() {
		displayInputParameters();
		
		const client = await newGrpcConnection();
		
		const gateway = connect({
			client,
			identity: await newIdentity(),
			signer: await newSigner(),
			hash: hash.sha256,
			evaluateOptions: () => ({ deadline: Date.now() + 5000 }),
			endorseOptions: () => ({ deadline: Date.now() + 15000 }),
			submitOptions: () => ({ deadline: Date.now() + 5000 }),
			commitStatusOptions: () => ({ deadline: Date.now() + 60000 }),
		});
		
		try {
			const network = gateway.getNetwork(channelName);
			const contract = network.getContract(chaincodeName);
			
			await initLedger(contract);
			await getAllAssets(contract);
			await createAsset(contract);
			await transferAssetAsync(contract);
			await readAssetByID(contract);
			await updateNonExistentAsset(contract);
		} finally {
			gateway.close();
			client.close();
		}
	}
	
	main().catch((error) => {
		console.error('** FAILED to run the application:', error);
		process.exitCode = 1;
	});
\end{lstlisting}


\section{Hyperledger Fabric Deployment Instructions}
\label{sec:deployment}

The following steps outline the process for deploying and testing the fish supply chain smart contract on Hyperledger Fabric using Google Cloud Platform.

\subsection{Environment Setup}
\begin{enumerate}
	\item \textbf{Open GCP and access the VM instance:}
	\begin{itemize}
		\item Navigate to Console $\rightarrow$ Compute Engine $\rightarrow$ VM instances $\rightarrow$ start $\rightarrow$ click SSH
		\item Alternatively: Virtual Machine $\rightarrow$ start $\rightarrow$ instance $\rightarrow$ SSH
	\end{itemize}
	
	\item \textbf{Connect to the instance:}
	\begin{lstlisting}[style=appendixstyle, caption={Connect to GCP VM via \texttt{gcloud}}, label={lst:connect-gcp}]
		gcloud compute ssh instance-20250322-102900 --zone=us-central1-c
	\end{lstlisting}
	
	
	
	\item \textbf{Navigate to the test network directory:}
	\begin{lstlisting}[style=appendixstyle, caption={Navigate to Compose Directory}, label={lst:navigate-compose}]
		cd ~/fabric-samples/test-network/compose
	\end{lstlisting}
\end{enumerate}

\subsection{Network Startup and Smart Contract Deployment}
\begin{enumerate}
	\item \textbf{Start the Hyperledger Fabric network:}
	\begin{itemize}
		\begin{lstlisting}[style=appendixstyle, caption={Start Fabric Network}, label={lst:start-network}]
			sudo docker-compose -f compose-test-net.yaml start
		\end{lstlisting}
		
		\item \textbf{Deploy the chaincode:}
		\begin{lstlisting}[style=appendixstyle, caption={Deploy Chaincode}, label={lst:deploy-chaincode}]
			cd ../
			./network.sh deployCC -ccn basic -ccp ../asset-transfer-basic/chaincode-go -ccl go
		\end{lstlisting}
		
		\item \textbf{Set environment path variables:}
		\begin{lstlisting}[style=appendixstyle, caption={Path Environment Variables}, label={lst:path-vars}]
			export PATH=${PWD}/../bin:$PATH
			export FABRIC_CFG_PATH=$PWD/../config/
		\end{lstlisting}
	
	\item \textbf{Configure organization environment variables:}
	\begin{lstlisting}[style=appendixstyle, caption={Org1 Environment Configuration}, label={lst:org-vars}]
			# Environment variables for Org1
			export CORE_PEER_TLS_ENABLED=true
			export CORE_PEER_LOCALMSPID="Org1MSP"
			export CORE_PEER_TLS_ROOTCERT_FILE=${PWD}/organizations/peerOrganizations/org1.example.com/peers/peer0.org1.example.com/tls/ca.crt
			export CORE_PEER_MSPCONFIGPATH=${PWD}/organizations/peerOrganizations/org1.example.com/users/Admin@org1.example.com/msp
			export CORE_PEER_ADDRESS=localhost:7051
	
	\end{lstlisting}
	\end{itemize}
\end{enumerate}
\subsection{Testing the Smart Contract}
\begin{enumerate}
	\item \textbf{Initialize the ledger:}
	\begin{lstlisting}[style=appendixstyle, breaklines=true, caption={Invoke InitLedger}, label={lst:init-ledger}]
		peer chaincode invoke -o localhost:7050 --ordererTLSHostnameOverride orderer.example.com --tls --cafile "${PWD}/organizations/ordererOrganizations/example.com/orderers/orderer.example.com/msp/tlscacerts/tlsca.example.com-cert.pem" -C mychannel -n basic --peerAddresses localhost:7051 --tlsRootCertFiles "${PWD}/organizations/peerOrganizations/org1.example.com/peers/peer0.org1.example.com/tls/ca.crt" --peerAddresses localhost:9051 --tlsRootCertFiles "${PWD}/organizations/peerOrganizations/org2.example.com/peers/peer0.org2.example.com/tls/ca.crt" -c '{"function":"InitLedger","Args":[]}'
	\end{lstlisting}
	
	\item \textbf{Query assets:}
	\begin{lstlisting}[style=appendixstyle, caption={Query Fish Asset}, label={lst:query-asset}]
		# Query a specific fish asset
		peer chaincode query -C mychannel -n basic -c '{"Args":["ReadAsset","tuna1"]}'
		
		# Query all assets in the ledger
		peer chaincode query -C mychannel -n basic -c '{"Args":["GetAllAssets"]}'
	\end{lstlisting}
	
	\item \textbf{Shut down the network:}
	\begin{lstlisting}[style=appendixstyle, caption={Stop Fabric Network}, label={lst:stop-network}]
		sudo docker-compose -f compose-test-net.yaml stop
	\end{lstlisting}
\end{enumerate}

\subsection{Important Notes}
\begin{itemize}
	\item Ensure proper network connectivity when working with Google Cloud Platform.
	\item The environment variables must be set correctly for the organization context.
	\item TLS certificates are required for secure communication between nodes.
	\item The commands assume the blockchain network is already provisioned on GCP.
\end{itemize}


% Save the file you want to include in PDF format.
% Uncomment the commant below specifying the correct appendix file. 
%\includepdf[pages=-, scale = 0.9, pagecommand={}, offset = -30 0]{appendixA.pdf}

