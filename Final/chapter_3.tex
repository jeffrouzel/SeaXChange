%   Filename    : chapter_4.tex 
\chapter{Research Methodology}
This chapter outlines a clear and detailed description of the research methods and processes used in the development and evaluation of SeaXChange: A Blockchain Driven App for Tuna Supply Chain Management. The algorithms, systems, theories, framework and models are described in detail in which this chapter establishes the foundation of this study .This chapter also explains the data collection method used ensuring the validity and reliability of the results.In addition, the chapter discusses the considerations and potential limitations of this study. Overall, this will serve as a guide for the readers in understanding the structured process of developing the SeaXChange.

\section{Software Development}
Scrum is a framework within the Agile development that prioritizes flexibility. It is an iterative software development approach that lets a project be broken down into phases and emphasizes continuous improvements. For this study, the researchers opted to use Scrum  because it involved many stakeholders and it operated in a ever-changing environment. Scrum allowed the team to adapt to new requirements through structured sprint planning, weekly reports, and sprint reviews, ensuring continuous alignment with project goals.

\section{Research Activities}
For this study, the researchers opted for interviews because it enabled in-depth exploration of stakeholder perspectives and experiences. The identified fisher and supplier client interface was tested within the perimeters of Barangay Sapa, Miagao, Iloilo, Philippines. The identified retailer testers were the vendors who reside in Barangay Mat-y and Barangay Sapa in Miagao. The identified consumer testers were situated in Miagao. The specific choice of these stakeholders and locations within Miagao was made to align with the study's defined scope, which concentrates on applying blockchain technology to enhance traceability and transparency within the small-scale tuna supply chain and localized market in the specified barangay of Miagao, Iloilo. This focus includes examining the traceability of whole, small-sized caught tuna products from capture through the local supply chain to consumers within this area. This localized approach facilitated a practical assessment of the SeaXChange application's potential effectiveness and user experience within a specific operational context.

\subsection{Feedback Collection Method}
The feedback was collected through a combination of structured interviews and assessment forms. Participants were asked to complete an assessment rubric evaluating the SeaXChange app across key criteria such as functionality, end-user needs, performance, usability, ease of use and feasibility. Moreover, follow-up interviews were conducted to gather deeper qualitative insights and obtain suggestions for system improvement.

The feedback gathered from fishermen, suppliers and retailers, and consumers was analyzed based on the SeaXChange assessment rubric, which evaluated six major categories: Functionality, End-user Needs, Performance, Usability, Ease of Use and Feasibility. The collected data were analyzed using descriptive statistics, through the computation of mean scores for each assessment criterion. These mean values were used to summarize stakeholder perceptions of the system. Mean ratings were calculated based on the 1-5 Likert Scale where 1 = Poor and 5 = Very Good.

\subsection{Data Gathering}
\begin{itemize}
	\item \textbf{Primary Data:} 
	\begin{itemize}
		\item Stakeholder(Fishermen, Supplier, Retailers, and Consumers) interviews were conducted to identify the use-case and user requirements, interface usability, and adoption challenges.
		\item Observations were made of existing tuna supply chain processes in local settings.
	\end{itemize}
	\item \textbf{Secondary Data:} 
	\begin{itemize}
		\item Literature review was conducted on blockchain applications in supply chain management and product traceability.
		\item Industry reports and regulatory documents related to tuna fishing and supply chain operations.
	\end{itemize}
\end{itemize}

\subsection{Designing and Developing the System}

\begin{enumerate}
	\item \textbf{Software Development Methodology:} The project followed a Scrum framework to ensure continuous iteration, stakeholder involvement, and flexibility in adapting to feedback.
	\item \textbf{Technology Stack:} 
	\begin{itemize}
		\item Front-end Development: Used React for creating a secure and user-friendly interface for stakeholders, prioritizing simple and responsive user-interface.
		\item Back-end Development: Used Node.js along with Express for managing back-end processes and API integration. Express is a flexible we application framework for Node.js used to build APIs for web applications. Docker for containerization of the project and Window Subsystem for Linux (Ubuntu as the Linux distribution) for setting up the network.
		\item Cloud Infrastructure: Used Google Cloud to host backend services and manage the databases, where the app could be accessed globally. It also ensured the app could scale smoothly as more data and users were added.  
		\item Blockchain Framework: Used Go language for developing smart contracts and providing an immutable ledger for transaction data.
		\item Database for Accounts: Used Firebase managing user accounts and authentication.
	\end{itemize}
	\item \textbf{Blockchain Development Platform:} 
	\begin{itemize}
		\item Used Hyperledger Fabric for its permissioned nature and scalable architecture.
		\item The open-sourced resources and timely updates of Hyperledger Fabric components is ideal for creating a distributed ledger for tuna supply chain.
	\end{itemize}
\end{enumerate}

\subsection{Implementing Algorithms and Services}
The system for this study is built on top of a Hyperledger Fabric project, it also utilized combinations of algorithms to facilitate the work flow of data or asset as well as ensuring high security with encryption and decryption configuration techniques. 
\begin{enumerate}
	\item \textbf{Consensus Algorithm} \\The project followed Raft(Leader-based consensus) for handling organizations or nodes. Raft is intended for managing a replicated log in a blockchain network. Raft is a Crash Fault Tolerant (CFT) protocol, is designed to handle non-malicious node failures (e.g., hardware crashes, network issues) In Raft, one node is elected as the leader, and it coordinates the ordering of transactions (Xu et al, 2022) \nocite{method-1}. The leader replicates log entries (transactions) to follower nodes, ensuring consistency across the network. 
	
	\item \textbf{Cryptographic Algorithm} \\The project employed several cryptographic algorithms to ensure security and privacy. These cryptographic data served as digital signatures and identity verification for the project. ECDSA (Elliptic Curve Digital Signature Algorithm) was used for generating digital signatures while X.509 certificates are intended for identity management and authentication of participants (Anitha \& Sankarasubramanian, n.d.) \nocite{method-2}. For the encryption, AES (Advanced Encryption Standard) was used for encrypting data at rest and in transit. TLS (Transport Layer Security) secured communication between network nodes. SHA-256 (Secure Hash Algorithm-256) ensured data integrity by generating unique hashes for blocks and transactions.
	
	\item \textbf{Membership Service} \\The implementation of the Membership Service Provider (MSP) requirement involved a set of folders added to the network configuration. These folders defined an organization both internally, by specifying its administrators, and externally, by enabling other organizations to verify the authority of entities attempting specific actions. While Certificate Authorities (CAs) are responsible for generating the certificates that represent identities, the MSP included a list of permitted identities. The MSP specified which Root CAs and Intermediate CAs are authorized to define members of a trust domain. This was achieved by either listing the identities of their members or identifying the CAs allowed to issue valid identities for those members.
	
	\item \textbf{Ordering Service} \\ The ordering service in this study played a crucial role in maintaining the integrity and functionality of the blockchain network. Its primary responsibilities included ensuring that transactions are processed in the correct sequence (transaction ordering), grouping transactions into blocks based on configurable parameters like size or timeout (block creation), and distributing these ordered blocks to peers for validation and commitment (block distribution) (Nassar et al, 2024)\nocite{method-3}. Additionally, the ordering service provided fault tolerance to ensure the network remains operational even in the presence of node failures through Raft.
	
	\item \textbf{Endorsement Policy} \\Fabric employs endorsement policies to specify which peers must validate a transaction before it's committed. The algorithm involved multi-signature schemes where a transaction is valid if it receives endorsements from the required peers as per the policy.
	
	\item \textbf{Chaincode (Smart Contract)} \\The handling and flow of business logic agreed to by members of the tuna supply chain in the blockchain network is executed by a chaincode or smart contract. The chaincode of the app was written in Go language. Docker container was used for enabling the chaincode to securely run along with the overall hyperledger fabric configurations. Chaincode initializes and manages ledger state through transactions submitted by applications (Hyperledger Fabric Documentation, 2024) \nocite{method-4}. The chaincode followed the object-oriented paradigm for creating classes and objects necessary for the tuna supply chain.
	
	
\end{enumerate}

\subsection{Modeling the System Architecture}
The system architecture of the project were consisted of many nodes that communicated with each other. The chaincode enabled the system to run algorithms, particularly, holding state and ledger data, and executes transactions such as asset transfer in the tuna supply chain. 

\begin{itemize}
	\item \textbf{Blockchain Architecture}
	\\The project involved peer, ordering services, ledger, and client application to perform various transaction such as tracing the origin and the stop points of a tuna asset. Peers are nodes in the blockchain network that maintained a copy of the distributed ledger and execute chaincode (smart contracts). The ordering service is the central component of the blockchain for ordering transactions and creating blocks to distribute to peers through consensus mechanism. The ledger is the immutable record of all transaction in the tuna supply chain network, stored across all peers. The client application is the interface through which users or tuna supply chain participants interact with the blockchain network.
	
	\begin{figure}[H]
		\centering
		\includegraphics[width=0.8\textwidth]{SeaXChange_model.png}
		\caption{Blockchain Architecture of SeaXChange}
		\label{fig:blockchain_model}
	\end{figure}
	
	\item \textbf{Overall System Architecture}
	\\The overall system architecture contains a web application built with Next.js for the frontend, utilizing Firebase for user authentication and account management. The application follows a role-based access model (Fisher, Supplier, Retailer, Consumer) where each role has specific permissions for interacting with tuna assets in the supply chain. The backend runs on Google Cloud Platform, consisting of an Express.js API that interfaces with a Hyperledger Fabric blockchain network (containerized in Docker) which stores and manages the immutable record of tuna assets and their transfers between supply chain participants. This architecture enables secure tracking of tuna from creation by fishers through the supply chain to consumers, with appropriate viewing and transfer capabilities assigned to each role in the process.
	
	\begin{figure} [H]
		\centering
		\includegraphics[width=0.8\textwidth]{overall-system.png}
		\caption{Overall System Architecture of SeaXChange}
	\end{figure}
	
	
	
	\item \textbf{Use Case}
	\\The use case shows the outline on how the user will interact with the SeaXChange App. It followed the major stages or participants in the tuna supply chain. 
	\begin{enumerate}
		\item \textbf{Fisher}
		\\- Encodes tuna I.D. of fish.
		\\- Encodes the date when the fish was captured.
		\\- Encodes the location where the fish was captured.
		\\- Encodes the fishing method used.
		\\- Query the origin and exchange of the tuna asset.
		
		\item \textbf{Supplier}
		\\- Encodes when the product was transferred from fisher to supplier.
		\\- Query the origin and exchange of the tuna asset.
		\\- Generate supplier's location during retrieval of tuna asset.
		
		\item \textbf{Retailer}
		\\- Encodes when the product was retrieved from the supplier or another retailer.
		\\- Query the origin and exchange of the tuna asset.
		\\- Generate retailer's location during retrieval of tuna asset.
		
		\item \textbf{Consumer}
		\\- Retrieve data from retailer.
		\\- Query the origin and exchange of the tuna asset.
		
		\begin{figure}[h]  
			\centering
			\includegraphics[width=0.8\textwidth]{seaxchange.png}
			\caption{Use case diagram for SeaXChange.}
			\label{fig:usecase}  
		\end{figure}
		
		There are four (4) types of users that will use the app. The first user type is the Fisher, which will be the starting point of the blockchain. It will encode the catch details of a tuna product such as the date of capture, location, and fishing method. The second user type is the Supplier, which will encode when the product was transferred from the fisher to the supplier, as well as generate their location during the retrieval of the tuna asset. The third type is the Retailer, which will encode when the product was transferred from the supplier to the retailer or in the case of multiple retailers, from the previous retailer to the current retailer, their location is also generated during the retrieval of the tuna asset. Lastly, the Consumers, which can only query the origin and exchange of tuna assets.
		
	\end{enumerate}
\end{itemize}